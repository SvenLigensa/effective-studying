\documentclass{ercisbeamer}

\title{Variation \& Interleaving}
\subtitle{Effective Studying}
\author{Sven Ligensa}
\institute{European Research Center for Information Systems (ERCIS)}
\date{\today}


\begin{document}

\setbgimage{00_resources/jungle_brain}
\begin{frame}
    \begin{tbox}
        \titlepage
    \end{tbox}
\end{frame}
\setbgimage{}

\begin{frame}{Contents}
    \tableofcontents
\end{frame}

\setbgimage{09_resources/Variation_50}
\section{Overview}
\begin{frame}{Overview}
    \begin{tbox}
        \begin{itemize}
            \item \red{\textbf{Variation}}: \red{Vary the exercises you engage in}
            \begin{itemize}
                \item On level of individual topic
            \end{itemize}
            \item \red{\textbf{Interleaving}}: \red{Interleave practice of multiple topics/subjects/skills}
            \begin{itemize}
                \item On level of multiple topics
                \item Switch \red{before} each practice is complete
                \item Alternate between different problems that need different solutions $\Rightarrow$ Need to recognize problem type first, then select right solution
            \end{itemize}
            \item Antipattern: \negative{Blocked} practice
            \begin{itemize}
                \item Still widespread in school and university
            \end{itemize}
            \item Analogy: Similar to how computers learn (Deep Learning)
            \begin{itemize}
                \item Variability of instances \grey{(e.g. Data Augmentation $\rightarrow$ Variation)} and concepts \grey{(Shuffling + Multiple epochs $\rightarrow$ Interleaving)} helps the model to generalize
            \end{itemize}
        \end{itemize}
    \end{tbox}
\end{frame}
\setbgimage{}

\section{Advantages}
\begin{frame}{Advantages}
    \begin{itemize}
        \item \positive{Versatility} of learning
        \begin{itemize}
            \item Finding commonalities/differences between different problems
            \item \positive{Broader understanding}
            \item \positive{Better assessment of context} and \positive{selecting right solution}
            \item \positive{Transfer of learning} between situations
        \end{itemize}
        \item Interleaving further implements Spacing $\Rightarrow$ Retrieval is harder
    \end{itemize}
\end{frame}

\section{Implementation}
\begin{frame}{Implementation}
    \begin{itemize}
        \item Vary the exercises $\rightarrow$ E.g. \red{Mock exams}
        \item \red{Shuffle} your flashcards
        \item Create \red{study plan} for which topic to study when \grey{(needs a bit more foresight)}
    \end{itemize}
\end{frame}

\section{Now You!}
\begin{frame}{Now You!}
    \begin{itemize}
        \item \emph{How exactly do you want to implement Variation and Interleaving into your study routine?}
        \item \emph{Do you want to create a study plan for the next examination phase?}
    \end{itemize}
\end{frame}

\section*{Outlook}
\begin{frame}{Outlook}
    \begin{enumerate}
        \item \positive{Introduction}
        \vspace{.5em}
        \item \positive{Illusions of Knowing}
        \item \positive{Understanding the Brain}
        \item \positive{Learning}
        \item \positive{Desirable Difficulties}
        \item \positive{Effective vs. Ineffective Learning Strategies}
        \vspace{.5em}
        \item \positive{Retrieval}
        \item \positive{Spacing}
        \item \positive{Variation and Interleaving}
        \item Next: \red{Mental Models}
        \item \grey{Memory Cues}
    \end{enumerate}
\end{frame}

\thankyou{Happy Learning!}{sven.ligensa@uni-muenster.de}

\sources

\end{document}
