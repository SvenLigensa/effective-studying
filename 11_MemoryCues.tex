\documentclass{ercisbeamer}

\title{Memory Cues}
\subtitle{Effective Studying}
\author{Sven Ligensa}
\institute{European Research Center for Information Systems (ERCIS)}
\date{\today}


\begin{document}

\setbgimage{00_resources/jungle_brain}
\begin{frame}
    \begin{tbox}
        \titlepage
    \end{tbox}
\end{frame}
\setbgimage{}

\begin{frame}{Contents}
    \tableofcontents
\end{frame}

\setbgimage{11_resources/massive_information}
\section{Overview}
\begin{frame}{Overview}
    \begin{tbox}
        \begin{itemize}
            \item Something with \red{\textbf{familiar structure to link information to}}
            \item Mental framework to store and later recall (large amounts of) information
            \item Organize what is already learned: \red{Understanding} first, \red{remembering} second!
            \begin{itemize}
                \item Should \red{\textbf{not}} be used to skip the understanding phase
            \end{itemize} 
            \item After repeated usage, not needed anymore
            \item \emph{Can you think of an example?} \pause
            $\rightarrow$ \red{P}lease \red{D}o \red{N}ot \red{T}hrow \red{S}ausage \red{P}izza \red{A}way \grey{(Seven layers of ISO OSI model)}
        \end{itemize}
    \end{tbox}
\end{frame}
\setbgimage{}

\section{Advantages}
\begin{frame}{Advantages}
    \begin{itemize}
        \item Commit \positive{huge} amounts of information to memory that have \positive{little to no connection}
    \end{itemize}
\end{frame}

\setbgimage{11_resources/memory_palace}
\section{Implementation}
\begin{frame}{Implementation}
    \begin{tbox}
        \begin{itemize}
            \item \red{Mnemonic Device}: Associating information with something that is easier to remember
            \item \red{Memory Palace}
            \begin{itemize}
                \item Based on method of loci: Associate mental images with series of physical locations (space familiar to you)
                \item Humans remember pictures more easily than words
                \item Associate vivid mental images with verbal/abstract material
                \item Images cue memories
                \item E.g. outline a speech in a series of sketches
            \end{itemize}
            \item \red{Rhyme schemes}: Peg method (peg = ``Klammer'') to remember lists
            \begin{itemize}
                \item E.g. song you know well
            \end{itemize}
        \end{itemize}
    \end{tbox}
\end{frame}
\setbgimage{}

\section{Now You!}
\begin{frame}{Now You!}
    \begin{itemize}
        \item \emph{Which of your courses benefit most from using Memory Cues? \grey{I.e. depend more on learning by heart than understanding and deducing information.}}
        \item \emph{What can be effective ways to come up with creative cues?}
        \item \emph{How can you commit them to memory?}
    \end{itemize}
\end{frame}

\chapteroverview{11}

\thankyou{Happy Learning!}{sven.ligensa@uni-muenster.de}

\sources

\end{document}
